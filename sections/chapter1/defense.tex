\chapter{Mitigation}
\label{s:mit}

In this section, we propose mitigation methods for the presented attacks. We
first enumerate precautionary suggestions that require no changes to the current
\eos system. We then discuss several design changes to the \eos system to remove
the root causes of the attacks presented herein.


\section{Preventative measures against attacks}
\label{ss:simple}

The \TESER, \SCPDOS, and \TOCTOU attacks can be easily prevented if \SCPs and
users carefully inspect their \SC source code as well as the \SCPs.
%
For the \TESER attack, an \SCP should verify if the code exists in their \SC that checks the SC's callers in a manner that they cannot abuse.
%
For example, one can add a permission check at the beginning of the \SC, so that
only authorized users can use the \SC.
%
On the other hand, to prevent the \SCPDOS attack, \SCPs should carefully audit their 
\SCs to determine whether there exists code that mistakenly uses their \ram.
%
In the case of the \TOCTOU attack, a \user should give his/her permission code 
only to the fully trusted \SCPs.

These simple preventions seem to be textbook examples, and none of them requires
any change to the underlying \PLATFORM architecture. To render the system more
secure, however, we do believe that the system must do its best effort to address
threats from every entity in the system.
%
For instance, in case of the \TESER attack, the \eos system can improve its security by
adding extra security components to it.
%
This approach is not uncommon as one of the largest systems, a Linux kernel,
adopted a similar concept to alleviate the threats derived from mistakenly
omitted permission checks~\cite{mcclure2009hacking}.
%
In the following subsections, we discuss potential changes to the \eos system to
fundamentally eliminate the root causes of the attacks presented above.

\section{Shifting the payment responsibility}
\label{ss:shifting}

As described in~\S\ref{label_teser}, in the \TESER attack, a malicious user intentionally
triggers a target SC to deplete the \cpu of its SCP. Of course, an SCP can
prevent this by carefully inspecting the code of their SC. However, this attack
stems from the current design that encourages \SCPs to pay for the execution of
their \SCs originating from user transactions.

In the current \eos system, to impose the transaction cost on a specific user,
an \SCP has to specify the user as authorized in the \SC, and this process
requires the \code permission from the user. In general, users are not normally
willing to grant this permission code since anyone with the code is able to
control the user account.
%
To avoid this, an SCP is supposed to create an SC with his/her own permission
code that imposes the transaction cost of executing the SC onto the SCP
himself/herself so that any user can be pleasant for the execution of the SC
without providing the permission or paying the cost.
%
Therefore, our design change suggestion removes the constraints of SCPs, thus
enabling them to avoid using their own resources, which also eliminates the
target of the attacker.

One way to address the \TESER attack could be to redesign the \eos system so that a user
who initiates a transaction should pay the transaction cost without having to
provide the \code permission.
%
It is quite intuitive to impose a transaction cost on the user who actually
wants to run the SC.
%
This fundamentally eliminates the liability of checking the permission of the
SC, as well as the benefit of the attack---thus, the adversary would no longer be
motivated to commit it.

On the other hand, this mitigation increases the possibility of opposite
cases in which a malicious SC could deplete the resources of a user because it 
does not need the permission code of the user.
%
Even when the user chooses to run a malicious SC, the system should be
responsible for alleviating the severity of the threat.
%
However, as the current \eos system supports limiting the maximum resource use
for running an SC, this can be mitigated as well~\cite{resourcemaxusage}.

\section{Fine-grained permission control}
\label{ss:fine}

The root cause of the \TOCTOU attack is a user giving his/her permission code
(\code) to an SCP in the belief that the SCP will not abuse this permission.
This, in fact, can be considered the fault of the user as s/he gives permission
with full awareness of the significance of their action. Nevertheless, the \eos
design can be changed to protect users from these kinds of threats.

Currently, the \eos permission system pursues an all-or-nothing
approach. That is, a user can make only one kind of action regarding his/her
permission---either give or revoke it.
%
This fundamentally facilitates threats on assets in \eos, including the \TOCTOU
attack.
%
The \eos developers are also aware of this issue and have proposed pinning
the permission into a specific version of the target SC. However, this still remains
an unresolved issue~\cite{EOSEOSIOCODE}.
%
Nevertheless, we have proven the feasibility of this threat, as shown
in~\S\ref{label_toctou}, and here we suggest possible mitigation that can eradicate
such threats.

We propose a fine-grained permission control, including the permission pinning 
described above. It enables a user to set a specific period or a specific
version that the permission depends on. As a result, in case of the 
\TOCTOU attack, even though a malicious SCP replaces a benign SC with a 
fraudulent one, the given permission would automatically lose its validity. 
Moreover, the expiration date of the permission can reduce the feasibility of 
the threats occurring because a user might forget to revoke permission.
%
Therefore, the fine-grained permission control can help to mitigate several
threats to the \eos assets.
