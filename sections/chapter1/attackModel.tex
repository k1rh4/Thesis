\chapter{Attack Model}

We assume an \eos adversary who utilizes the same functionalities of the \eos
system as \SCPs and their service users do.
%
That is, the adversary is able to implement and manage her own SCs, or to
execute other SCs.
%
The adversary can also entice victims to send a transaction that triggers the
execution of a crafted SC.
%
Furthermore, the adversary is capable of updating the crafted \SC anytime to
leverage the misplaced trust of the victims who only examined the previous SC
before the update.
One goal of the adversary is to deplete available resources of \SCPs, such as
\cpu, \net, or \ram, thus causing the denial of service (DoS) for these \SCPs.
The adversary also aims to cause delays in block generation in BPs,
because such delays significantly undermine the service availability of the \eos
system.
%
Consider SCs for stock trading or banking services, which demand a short latency
for task completion. In such services, any system delay can potentially
lead to a significant financial loss as transactions that should be
processed within a short time cannot be processed.
%
Furthermore, the adversary could also plunder the resources of the victims and
demands a ransom.

Note that our adversary model is similar to those used in the previous studies
that exploit \SC bugs~\cite{\smartcontractPapers} in \eth or cause double 
spending in \btc with spam transactions~\cite{blockproblem}, except that the 
adversary is able to update the crafted \eos \SC.

