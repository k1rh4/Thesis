\chapter{Related work}

%To the best of our knowledge, no research has analyzed the risk of \PLATFORM
%architecture. Several studies focused on inspecting the \ETHER's \SC
%vulnerabilities and blockchains, and There are some \PLATFORM's vulnerabilities
%on online. The system architecture of \PLATFORM's features across different
%environments with other blockchain platforms

%\subsection{블록체인에서의 취약점 연구 사례}
%\ETAL
% [합의 알고리즘] I. Eyal \ETAL \cite{eyal2018majority}: Selfishi mining [합의
% 알고리즘] I. Eyal, "The miner's dilemma", IEEE S&P'15: BWH 공격 얘기.. [합의
% 알고리즘] Y. Kwon, D. Kim, Y. Son, E. Vasserman, Y. Kim, "Be Selfish and Avoid
% Dilemmas:Fork After Withholding (FAW) Attacks on Bitcoin", ACM CCS'17: PAW
% 공격 얘기..

% [블록체인을 운영하는 노드를 공격함.] E. Heilman, A. Kendler, A. Zohar, S.
% Goldberg, "Eclipse Attacks on Bitcoin's Peer-to-Peer Network", USENIX Sec'15

\section{Security analysis of blockchain systems}

Many previous studies have investigated the security of blockchain
systems~\cite{\bcPapers}. Most of them focused on analyzing the two blockchain
systems; \btc and \eth.
%Here, we introduce a few of them.
%
There is a large body of work on \btc attacks. Researchers have analyzed 
double spending~\cite{karame2012double}, network-related attacks~
\cite{heilman2015eclipse, apostolaki2017hijacking}, illicit transactions~
\cite{son2019ndss}, transaction malleability~\cite{decker2014bitcoin,
andrychowicz2015malleability}, and selfish mining~\cite{\miningpoolPapers}.
%
Eyal~\etal~\cite{eyal2018majority} introduced a selfish mining problem in
which rational \btc miners collude to control the entire \btc system, including 
taking over the mining profit.
%
Heilman~\etal~\cite{heilman2015eclipse} demonstrated the eclipse attack on the 
\btc network where  an adversary can fabricate \btc mining power by spoofing the 
routing tables of other miners. They utilized this to conduct double spending attacks as well.
%
Kwon~\etal~\cite{kwon2017selfish} proposed the fork-after-withholding attack,
in which selfish miners secretly keep new blocks as long as possible to maximize
their profit.

Previous research has also investigated the security of \eth smart contracts~
\cite{\smartcontractAll}.
%
Luu~\etal~\cite{luu2016making} introduced an automatic bug finding tool for \eth
SCs. They aimed to discover bugs that allow colluding miners to intentionally 
reorder the execution of SCs or to change the timestamp of SCs.
%
They also presented the notorious re-entrancy bug, which allows an adversary to
continuously execute a target SC for various purposes.
%
Krupp~\etal~\cite{krupp2018teether} developed a tool that
automatically generates exploits for identified SC vulnerabilities, and
Kalra~\etal~\cite{kalra2018zeus} presented another tool that automatically
remediates a target \SC by inserting security checks.
%
%There have been other open-source tools~\cite{\smartcontractTools} as well.

Recently, researchers have started analyzing the security of other blockchain
systems, such as Ripple~\cite{rippleattack},
Monero~\cite{hinteregger2018empirical, moser2018empirical}, and
Stellar~\cite{kim2019is}. In addition, there have been studies on evaluating
various consensus algorithms~\cite{zhang2019lay}, and analyzing the decentrality
in permissionless blockchains~\cite{kwon2019impossibility}.

In line with this research, we analyzed the \eos system an identified new security concerns that are rooted in the unique characteristics of \eos and also described their attack scenarios.
%
We are planning to develop an analysis platform for \eos SCs, following the
previous papers~\cite{luu2016making, krupp2018teether}.


\section{Security analysis of \eos}

Even though there exist many studies on \btc and \eth, there have been only a few studies conducted on \eos. 
%
%In fact, we tried to find other academic papers on the security of \eos, but we were unable to find even one. Here, we introduce several bugs~\cite{\eosBugs} found in the industry.
We only found several bugs~\cite{\eosBugs} reported from the industry.
%
These bugs can be categorized as software bugs in the core \eos system
~\cite{\eosSoftwareBugs}, logical bugs in \eos SCs~\cite{\eosSCBugs}, and a
design bug~\cite{\eosDesignBugs}.

Chen~\etal~\cite{EOSARBITCODE} found an arbitrary code execution bug in \eos.
This bug is essentially caused by an integer overflow bug when executing an SC
that leads to out-of-bound buffer access.
%As a result, an adversary could take over the control of the nodes.
%
There are bugs from mis-implementation of SCs as well~\cite{\eosSCBugs}. One
example is that an adversary can exploit an SC that does not properly check the
caller of the SC, allowing multiple \eos SCs to be executed
subordinately~\cite{requireRecipient}.
%
The third category is a bug that stems from the \eos core design, allowing a
large number of transactions to delay the block creation time~\cite{eosNodeDoS}.
This appears to be similar to our \NODEDOS attack; however, they essentially
differ in that the \NODEDOS attack pauses BPs by exploiting their transaction
scheduling algorithm. Hence, it can effectively break the system down.

In this paper, we documented four new threats and attacks. Even though some of
them seem trivial,  their underlying causes are related to the \eos system
design. We also addressed specific design concerns with the current \eos system
and noted how fixing them would address potential threats to \eos.
