\chapter{Introduction} %Don't want chapter number for intro

Recently, \textit{cryptocurrency} has attracted a great deal of attention due
to its rapidly growing market cap. Bitcoin, often called the first
cryptocurrency, reached a market cap of over 100 billion USD in 2019, and more
than 2K cryptocurrencies have emerged worldwide, thus representing a combined
value of over 200 billion USD~\cite{coincap}. This surging interest in the
cryptocurrency has also attracted the attention of both industry and
academia for its underlying \textit{blockchain} system. Consequently, many
studies have been conducted to analyze and improve the core technology of each
blockchain system~\cite{\bcPapers}.

The security of the blockchain is another important issue for which vast
research has been conducted. These studies focused on analyzing double
spending~\cite{karame2012double}, network-related
attacks~\cite{heilman2015eclipse, apostolaki2017hijacking},
illicit transactions~\cite{son2019ndss}, transaction
malleability~\cite{decker2014bitcoin, andrychowicz2015malleability}, selfish
mining~\cite{\miningpoolPapers}, or smart contracts~\cite{\smartcontractAll}.
However, most of these studies have only focused on the two major
cryptocurrencies; \btc and \eth.

\eos is an emerging blockchain system, whose cryptocurrency, \EOS, possesses a
market cap of over seven billion USD~\cite{coincap}, making it one of the top
five cryptocurrencies worldwide.~\footnote{
We checked this at the time of the submission.}
%
There have been only a few studies on its security~\cite{\eosBugs}.
%
The \eos foundation has been incentivizing researchers through a bug bounty
program, but most of the bugs found so far are related to implementation, such
as software vulnerabilities that an adversary can exploit to perform
arbitrary code execution~\cite{\eosSoftwareBugs}, or logical bugs that cause a
smart contract to deviate from the objectives of its developers
~\cite{\eosSCBugs}. However, as the design of \eos is markedly different from
that of \btc or \eth in managing system resources, there may exist distinctive
and unexplored threats that stem from the resource management policies of the \eos
system.

To address the lack of research in this area, we chose \eos as the objective of
this study and analyzed its security.
%
We first studied the characteristics of the \eos system. Note that some parts
of the \eos whitepaper~\cite{EOSWHITEPAPER} are outdated or ambiguous; hence,
we had to manually analyze the source code to figure out its actual implemented
semantics.

To speed up the slow transaction rate of previous blockchain systems, \eos
adopted a \textit{Delegated Proof of Stake (DPoS)} algorithm~\cite{
larimer2014delegated}, which delegates the consensus on blocks and execution of
the smart contracts to the representative nodes, called Block Producers (\BPs).
%\kirhanote{EOS가 DPoS를 대표하는 코인이라는 말을 추가하면 어떨까요?}
This design decision of using DPoS entails problems of sharing computational
resources among a small number of \BPs. To effectively manage resources and
prevent resource abuse, \eos asks a participant to obtain a resource capacity,
such as computational power, network bandwidth, and storage~\cite{
EOSWHITEPAPER}.
%
Even though this new means of resource management can effectively support the
representative nodes in executing smart contracts, it raises new concerns about
added assets that can negatively impact the security of the
system~\cite{howard2006security}.


% Important of resource managements.
Any attack on resource managements that leads to a denial of service (DoS) is a
critical security threat. Many emerging blockchain systems strive to achieve 
high Transactions-Per-Second (TPS) rate, which enables the provision of a
short-latency service for their users and service providers. Incessant attacks
that successfully undermine the availability or TPS of blockchain services break
fundamental requirements that blockchain service users expect. For instance,
stock transaction services or payment services at retail stores demand a short
latency to process financial transactions. Exacerbating transaction time
directly affects the quality of these services, thus contributing to users
choosing other blockchain services.


Therefore, we investigated resource-related threats that exploit the design
architecture unique to \eos. We manifested the capabilities of smart contract
providers (SCPs) as well as their service users and defined an \eos adversary
who is able to conduct the same level of privileged operations as these
participants do. Under this adversary model, we identified four possible attack
scenarios: \NODEDOS, \TESER, \SCPDOS, and \TOCTOU attacks.
%
In these attacks, an adversary intentionally incapacitates the entire \eos system
with a crafted smart contract, an adversary depletes or drains the resources of
a smart contract provider so that this victim no longer provides the service,
and a malicious smart contract provider surreptitiously replaces the original smart
contract with another one to steal user assets.

In order to address these attacks, we analyze their causes and propose several
mitigations. These mitigations include simple preventative measures along with
suggestions for systematic changes of \eos. We believe that the proposed
mitigations can prevent the attacks described in this paper as well as potential
future threats.

In summary, our contributions are as follows:
\begin{itemize}
\item We present the \emph{block delay} attack, a novel attack that exploits
    a transaction scheduling policy of the \eos system. This attack is able
    to delay all transactions in the system, thus resulting in the DoS.
\item We introduce two new attack methods: \emph{\TESER and \SCPDOS} attacks.
    These attacks exploit resource management policies unique to the system,
    thus disrupting services from a target smart contract provider.
\item We present the \emph{\TOCTOU} attack scenario that abuses controversial
    design decisions of the system. The attack allows locking \ram resources
    of a target user, which enables an attacker to ask a ransom in exchange for
    releasing the locked \ram resources.
\item We demonstrate the feasibility of each attack and estimate the impact of
    each proposed attack scenario based on experiments conducted in a local \eos
    system.
\item We propose preventative measures as well as a redesign of the \eos system
    to mitigate the proposed and potential threats.
\end{itemize}

