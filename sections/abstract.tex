% ABSTRACT
\begin{abstract}
\EOS is a popular cryptocurrency, whose market cap is over seven billion USD.
Its ecosystem operates in the \PLATFORM system, which is devised to speed up the
slow transaction rate of previous blockchain technologies.
%
Whereas many previous studies have investigated the security issues of \btc and
\eth, the security of \PLATFORM has thus far drawn little attention despite
its popularity.
%
Even the studies that have addressed the security of EOS and its underlying
blockchain system mostly focused on implementational bugs in the core of
the \PLATFORM system or in smart contracts, rather than addressing the
fundamental problems stemming from the \eos design.

To address this void in the previous literature, we investigate the design
architecture of \eos. Based on this investigation, we introduce four attacks
whose root causes stem from the unique characteristics of \eos, including
intentionally slowing down the block creation time---which can disrupt the
essential functions of its blockchain and incapacitate the entire \eos system.
%
In addition, we find that an adversary can partially freeze the execution of
a target smart contract or maliciously consume all the resources of a target
user with crafted requests.
%
We report all the identified threats to the \eos foundation, one of which
is confirmed to be fatal.
%
Finally, we discuss possible mitigations against the proposed attacks.

\end{abstract} 

% KEYWORD
\begin{Engkeyword}
	EOS, EOS.IO, Resource, Security, Vulnerabilities
\end{Engkeyword}
\newpage

